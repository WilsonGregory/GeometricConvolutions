% Copyright 2022 the authors

\documentclass{article}
\usepackage[utf8]{inputenc}
\usepackage[letterpaper]{geometry} % because Overleaf is weird.
\PassOptionsToPackage{hyphens}{url}
\usepackage[hidelinks]{hyperref}
\usepackage{graphicx}
\usepackage{amsfonts}
\usepackage{amsmath}
\usepackage{amsthm}
\theoremstyle{definition}
\newtheorem{definition}{Definition}
%TODO: change N to N^3 in the appropriate locations
% define epsilon, they depend on the dimension


% typesetting adjustments
\makeatletter
\renewcommand\section{\@startsection {section}{1}{\z@}%
  {-3.25ex \@plus -1ex \@minus -.2ex}%
  {1.5ex \@plus .2ex}%
  {\raggedright\normalfont\large\bfseries}}
\makeatother
\setlength{\textwidth}{5.00in}
\setlength{\oddsidemargin}{3.25in}
\addtolength{\oddsidemargin}{-0.5\textwidth}
\setlength{\textheight}{9.40in}
\setlength{\topmargin}{-0.50in}
\setlength{\footskip}{0.25in}
\pagestyle{myheadings}
\markboth{foo}{Hogg \& Villar / Emulating n-body simulations with equivariant CNNs}
\newcommand{\figurerule}{\rule[1ex]{\textwidth}{0.2pt}}
\linespread{1.08}
\frenchspacing\sloppy\sloppypar\raggedbottom

\title{\bfseries%
Emulating cosmological n-body simulations with equivariant convolutional neural networks}
\author{Hogg, Villar, others}
\date{2022 April}

\begin{document}

\maketitle\thispagestyle{empty}

\section{Introduction}

Cosmological n-body simulations comprise the most computationally expensive part of most contemporary cosmological projects, missions, and experiments (CITE THINGS).
For this reason, accurate, fast emulators would be of enormous economic and scientific value.
This point has not gone unnoticed!
There are now quite a few successful projects implementing such emulators (CITE THINGS).

However, there are dangers involved in emulation of this kind.
HOGG: Epistemological and ontological problems!

The epistemological problems can be somewhat ameliorated (though not entirely solved) if the emulators are required to obey all of the exact symmetries that the n-body simulations themselves are expected to obey.
That is what motivates the present project.

HOGG: Summarize here how we are going to proceed, in human language.

\section{Problem setup}

For our purposes, a cosmological simulation is a set of $N^3$ particles $n$ with $1\leq n\leq N^3$, each of which has a (scalar) mass $m_n$, a (vector) position $q_n$, and a (vector) veloctiy $v_n$, and each of which is a function of time (or computed or stored at a set of times $t$).
The positions $q_n$ live in a cubical 3-torus of side length $L$.
In the dark-matter-only case, the simulation dynamics depend only on the masses, positions, velocities $\{m_n, q_n, v_n\}_{n=1}^N$, time $t$, Newton's constant $G$, and (hopefully weakly) the number $N^3$ and length $L$.

Because the simulations are designed to describe the evolution of the Universe, the particles begin in a nearly-uniform density state, usually very close to a uniform, torus-filling, cubical lattice.
The particles begin at a very early time $t$ with tiny (vector) offsets $\Delta q_n$ away from lattice sites $q^{(0)}_n$ such that $q_n = q^{(0)}_n + \Delta q_n$, and tiny (vector) velocities $v_n$.
In detail these positional offsets and velocities are closely related by something like a Zeldovich approximation.

The uniform, torus-filling, cubical lattice of $N^3$ points $q^{(0)}_n$ can be thought of as a 3-dimensional ``image''.
Each particle can be thought of as a ``pixel'' in that image.
At each snapshot of the simulation at time $t$, each pixel has an associated (vector) positional offset $\Delta q_n$, (vector) velocity $v_n$, and a few scalars ($m_n, t, G$ at least).
In addition to these native scalars, we can produce at each pixel $n$ three additional scalars $q_n^\top q_n, q_n^\top v_n, v_n^\top v_n$, which are the inner products of the vectors with themselves and each other.
At late times, the simulation shows almost no memory of the initial uniform, torus-filling, cubical lattice.
However, each particle has an identity $n$ that can be associated with its original lattice location $q^{(0)}_n$, so its properties can be thought of as being encoded in that original lattice-location pixel.
That is, this is a problem of scalars and vectors on a uniform, cubical lattice in a cubical 3-torus.

There are some exact and approximate symmetries to consider in this problem.
The approximate symmetries are the fundamental symmetries of classical physics, which include translation, rotation, and parity.
These symmetries are only approximate in the n-body simulation context, because there is a cubical 3-torus with an orientation and a size, and there is a cubical lattice inside that 3-torus.
Generators of the exact symmetries are the three torus-axis-aligned spatial rotations by $\pi/2$ and one reflection across one of the three axes of the torus.

\section{Linear geometric convolution operators}

We consider the orthogonal group $O(d)$ and $g\in O(d)$. For our purposes $d$ will be 2 or 3, but everything generalizes.  
We consider an image $A$ in a cubical $d$-torus with $N$ equally spaced pixels in each dimension ($N^d$ pixels total). Each pixel can contain tensors and pseudotensors (of which scalars, pseudoscalars, vectors, and pseudovectors are particular cases) that we define below.
Similarly, each linear geometric convolution operator $C$ will be a pixelized cubical image patch with $M$ pixels (with $M$ odd) in each dimension ($M^d$ pixels total) with a tensor or pseudotensor in each pixel, which acts as a linear pixel weight.

For some of the expressions below we will use Ricci and Levi-Civita summation notation (or Einstein summation notation).
In this notation, tensor products are written in component form, and repeated indices are summed over.
In this notation, the product of two 2-tensors (represented as two $d\times d$ matrices $A$ and $B$) is written as
\begin{equation}
    [A\, B]_{i,j} = [A]_{i,k}\,[B]_{k,j} ~,
\end{equation}
where $[A]_{i,k}$ (for example) is the $i,k$ element of matrix $A$, the repeated index $k$ is implicitly summed over the integers from 1 to $d$.
This notation works for tensor expressions of any order, provided that every index appears either exactly once (so it isn't summed over; in the above expression $i, j$ appear exactly once) or exactly twice (so it is summed over; in the above expression $k$ appears exactly twice). 

\begin{definition}[vectors, $k$-tensors, and the world]
Let $O(d)$ act on vectors in $\mathbb R^d$ via the canonical action.
It also acts on lists of vectors $(\mathbb R^d)^n$ by acting on each of the $n$ vectors by the same action.
It also acts on rank-1 $k$-tensors in $(\mathbb R^d)^{\otimes k}$ by $g\cdot (v_{1}\otimes\ldots \otimes v_k) = g\cdot v_1\otimes \ldots \otimes g\cdot v_k$ and extending linearly to higher-rank tensors made from sums of rank-1 tensors.
Let $W$ (aka ``the world'') be the comprehensive set of all vectors and tensors that exist.
Note that $O(d)$ can be seen as acting on the world.
\end{definition}

One consequence of the above defnitions is that 
$T\in (\mathbb R^d)^{\otimes k}$ is a $k$-tensor iff (in summation notation)
$[g\cdot T]_{i_1,\ldots, i_k} = [T]_{j_1,\ldots,j_k}\,[M(g)]_{i_1,j_1}\cdots[M(g)]_{i_k,j_k}$ for all $g\in O(d)$, where $[T]_{i_1, \ldots ,i_k} \in \mathbb R$ is a component of $T$, $[M(g)]_{i,j}\in\mathbb R$ is the $i,j$ element of the matrix representation of $g$, and all the $i_m$ and $j_m$ are indices in the range $1,\ldots,d$.
For example, a 2-tensor is defined by its transformation property
$[g\cdot T]_{i,j} = [T]_{k,\ell}\,[M(g)]_{k,i}\,[M(g)]_{\ell,j}$,
which, in normal matrix notation, is written as
$g\cdot T = M(g)\,T\,M(g)^\top$.

\begin{definition}[pseudovector and $k$-pseudotensor]
SOLE CAN WE SAY HERE that these are ``like a vector and like a tensor'' but there is an additonal $\det(g)$ coming in?
\end{definition}

\begin{definition}[scalar and pseudoscalar]
We say that $a\in \mathbb R$ is a $0$-tensor or scalar iff $a$ is an invariant of the action.
Namely, $a(W) = a(g\cdot W)$ for all $g\in O(d)$.
We say that
$a\in \mathbb R$ is a $0$-pseudotensor or pseudoscalar iff $a(g \cdot W) = \det(g)\,a(W)$ for all $g\in O(d)$,
where $\det(g)$ is the determinant of the matrix representation $M(g)$ of $g$.
\end{definition}

\begin{definition}[outer products of tensors]
If $a$ is a $k$-tensor and $b$ is a $k'$-tensor, then the outer product $a\otimes b$ is a $(k+k')$-tensor with $[a\otimes b]_{i_1,\ldots,i_{k+k'}} = [a]_{i_1,\ldots,i_k}\,[b]_{i_{k+1},\ldots,i_{k+k'}}$
Similarly, if $a$ is a $k$-pseudotensor and $b$ is a $k'$-pseudotensor, then $a\otimes b$ is a $(k+k')$-tensor.
If $a$ is a $k$-tensor and $b$ is a $k'$-pseudotensor, or if $a$ is a $k$-pseudotensor and $b$ is a $k'$-tensor, then $a\otimes b$ is a $(k+k')$-pseudotensor.
\end{definition}

\begin{definition}[linear geometric convolution]
Given an image $A$ of geometric objects with $N^d$ pixels, we define its linear convolution with a geometric filter $C$ as follows.
Each pixel $p$ (numbered $1\leq p\leq N^d$) of the image contains a geometric object $a_p$.
All of these $N^d$ geometric objects $a_p$ are of the same geometric type ($k$-tensor or $k$-pseudotensor).
The filter $C$ has side length $M\leq N$, where $M$ is an odd integer, such that it has $M^d$ total pixels.
Each pixel $m$ (numbered $1\leq m\leq M^d$) of the filter contains a geometric object $w_m$.
All of these $M^d$ geometric weights $w_m$ are of the same geometric type as well ($k'$-tensor or $k'$-pseudotensor).
For image pixel $p$ (with $1\leq p\leq N^d$) we take a $M^d$ patch of the image, centered at $p$ with contents $a(p)$ (where we assume that the image is in the $d$-torus, so every pixel centers a $M^d$ patch) and number its pixels $1\leq m\leq M^d$.
Then the convolution $A\star C$ of the image $A$, evaluated at the central pixel $p$ of the patch $P_p$ (containing pixel values $a(p)_m$ with $1\leq m\leq M^d$) and the filter $C$ is the sum of outer products
\begin{equation}
    (A\star C)_p := \sum_{m=1}^{M^d} a(p)_m \otimes w_m.
\end{equation}
\end{definition}

If the image pixels $a_m$ are $k$-tensors and the filter weights $w_m$ are $k'$-tensors, or if the image pixels are $k$-pseudotensors and the weights are $k'$-pseudotensors, then the output of the linear filter is an image of $(k+k')$-tensors.
If the image pixels are $k$-pseudotensors and the weights are $k'$-tensors, or if the image pixels are $k$-tensors and the weights are $k'$-pseudotensors, then the output is an image of $(k+k')$-pseudotensors.
This linear geometric convolution operation can be followed by contractions with Kroenecker deltas and Levi-Civita symbols to make lower-$k$, but still linear, objects.

Contractions of these objects are possible, including contractions that involve either the (even parity) Kroenecker delta $\delta_{ij}$ or the (odd parity) Levi-Civita symbol $\epsilon_{ijk}$ (in $d=3$ dimensions; $\epsilon_{ij}$ in $d=2$ dimensions; $\epsilon_{ijk\ell}$ in $d=4$ dimensions).
The Levi-Civita symbol in $d\geq 2$ spatial dimensions is the object with $d$ indices such that if the $d$ indices are not repeated and in cyclic order the value is $+1$ and if the $d$ indices are not repeated and in anti-cyclic order the value is $-1$, and it has the value $0$ in all other cases.
For an example of how the Levi-Civita symbol is used, in $d=3$ dimensions, the determinant of a $d\times d$ matrix $A$ is given by $\det(A) = [A]_{1,i}\,[A]_{2,j}\,[A]_{3,j}\,\epsilon_{ijk}$ in summation notation.
Contraction of a $k$-tensor (with $k\geq 2$) with one application of the Kroenecker delta returns a $(k-2)$-tensor.
Contraction of a $k$-pseudotensor (with $k\geq 2$) with one application of the Kroenecker delta returns a $(k-2)$-pseudotensor where $d$ is the spatial dimension.
Contraction of a $k$-tensor (with $k\geq (d-1)$) with one application of the Levi-Civita symbol returns a $(k-d+2)$-pseudotensor, because the contractions are structured with the last of the $d$ indices of the Levi-Civita symbol unpaired. SOLE DO WE NEED TO SAY MORE HERE?
Contraction of a $k$-pseudotensor (with $k\geq (d-1)$) with one application of the Levi-Civita symbol returns a $(k-d+2)$-tensor.

For one example of these contractions, the 2-tensor formed by the outer product of 1-tensors $a$ and $b$ can be contracted with the Kronecker delta to give the standard dot product $[a]_i\,[b]_j\,\delta_{ij}$, which is a 0-tensor or scalar.
For another example, the same 2-tensor can be contracted with the Levi-Civita symbol to give the standard cross product
$[a]_i\,[b]_j\,\epsilon_{ijk}$, which is a 1-pseudotensor or pseudovector.

With these definitions we can make geometric filters, take outer products, and contract to make a combinatorically large set of nonlinear geometric convolution operators on images made up of geometric objects.
But before we go nonlinear, we will go equivariant.

\section{Equivariant linear geometric convolution operators}

We will now create linear convolution operators that are group-equivariant, for groups of reflections, rotations, and translations (on the cubical torus).
We will construct these equivariant convolution operators by an averaging over group operators, followed by a matrix factorization (singular-value decomposition).

\begin{definition}[Equivariant convolution operator]
Foo and bar.
\end{definition}

Given a $k\times k \times k$ patch ($k$ odd), centered on a particular pixel, we can consider the convolution operator with a fixed $k\times k \times k$ filter. The filter in each pixel can contain a tensor or pseudotensor

HOGG: What is a linear convolution operator?

HOGG: What is a linear geometric convolution operator?

HOGG: What would it mean for a linear geometric convolution operator to be equivariant to all relevant groups?
It's naturally translation-equivariant!
But then how do we do reflections and rotations?
Be sure to do scalar, pseudoscalar, vector, pseudovector, 2-tensor, and 2-pseudotensor cases.

HOGG: How does the group act, on the image, and on the linear geometric convolution filter, and both?

\section{Equivariant nonlinear geometric convolutions}

\section{Construction of all equivariant convolution operators}

Group averaging and SVD to get all equivariant linear convolution operators.

Outer products and contractions to get all equivariant nonlinaer convolution operators.

Lots of pretty pictures.

\end{document}
